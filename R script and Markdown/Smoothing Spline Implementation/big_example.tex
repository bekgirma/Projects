% Options for packages loaded elsewhere
\PassOptionsToPackage{unicode}{hyperref}
\PassOptionsToPackage{hyphens}{url}
%
\documentclass[
  ignorenonframetext,
]{beamer}
\usepackage{pgfpages}
\setbeamertemplate{caption}[numbered]
\setbeamertemplate{caption label separator}{: }
\setbeamercolor{caption name}{fg=normal text.fg}
\beamertemplatenavigationsymbolsempty
% Prevent slide breaks in the middle of a paragraph
\widowpenalties 1 10000
\raggedbottom
\setbeamertemplate{part page}{
  \centering
  \begin{beamercolorbox}[sep=16pt,center]{part title}
    \usebeamerfont{part title}\insertpart\par
  \end{beamercolorbox}
}
\setbeamertemplate{section page}{
  \centering
  \begin{beamercolorbox}[sep=12pt,center]{part title}
    \usebeamerfont{section title}\insertsection\par
  \end{beamercolorbox}
}
\setbeamertemplate{subsection page}{
  \centering
  \begin{beamercolorbox}[sep=8pt,center]{part title}
    \usebeamerfont{subsection title}\insertsubsection\par
  \end{beamercolorbox}
}
\AtBeginPart{
  \frame{\partpage}
}
\AtBeginSection{
  \ifbibliography
  \else
    \frame{\sectionpage}
  \fi
}
\AtBeginSubsection{
  \frame{\subsectionpage}
}
\usepackage{amsmath,amssymb}
\usepackage{lmodern}
\usepackage{iftex}
\ifPDFTeX
  \usepackage[T1]{fontenc}
  \usepackage[utf8]{inputenc}
  \usepackage{textcomp} % provide euro and other symbols
\else % if luatex or xetex
  \usepackage{unicode-math}
  \defaultfontfeatures{Scale=MatchLowercase}
  \defaultfontfeatures[\rmfamily]{Ligatures=TeX,Scale=1}
\fi
% Use upquote if available, for straight quotes in verbatim environments
\IfFileExists{upquote.sty}{\usepackage{upquote}}{}
\IfFileExists{microtype.sty}{% use microtype if available
  \usepackage[]{microtype}
  \UseMicrotypeSet[protrusion]{basicmath} % disable protrusion for tt fonts
}{}
\makeatletter
\@ifundefined{KOMAClassName}{% if non-KOMA class
  \IfFileExists{parskip.sty}{%
    \usepackage{parskip}
  }{% else
    \setlength{\parindent}{0pt}
    \setlength{\parskip}{6pt plus 2pt minus 1pt}}
}{% if KOMA class
  \KOMAoptions{parskip=half}}
\makeatother
\usepackage{xcolor}
\newif\ifbibliography
\usepackage{color}
\usepackage{fancyvrb}
\newcommand{\VerbBar}{|}
\newcommand{\VERB}{\Verb[commandchars=\\\{\}]}
\DefineVerbatimEnvironment{Highlighting}{Verbatim}{commandchars=\\\{\}}
% Add ',fontsize=\small' for more characters per line
\usepackage{framed}
\definecolor{shadecolor}{RGB}{248,248,248}
\newenvironment{Shaded}{\begin{snugshade}}{\end{snugshade}}
\newcommand{\AlertTok}[1]{\textcolor[rgb]{0.94,0.16,0.16}{#1}}
\newcommand{\AnnotationTok}[1]{\textcolor[rgb]{0.56,0.35,0.01}{\textbf{\textit{#1}}}}
\newcommand{\AttributeTok}[1]{\textcolor[rgb]{0.77,0.63,0.00}{#1}}
\newcommand{\BaseNTok}[1]{\textcolor[rgb]{0.00,0.00,0.81}{#1}}
\newcommand{\BuiltInTok}[1]{#1}
\newcommand{\CharTok}[1]{\textcolor[rgb]{0.31,0.60,0.02}{#1}}
\newcommand{\CommentTok}[1]{\textcolor[rgb]{0.56,0.35,0.01}{\textit{#1}}}
\newcommand{\CommentVarTok}[1]{\textcolor[rgb]{0.56,0.35,0.01}{\textbf{\textit{#1}}}}
\newcommand{\ConstantTok}[1]{\textcolor[rgb]{0.00,0.00,0.00}{#1}}
\newcommand{\ControlFlowTok}[1]{\textcolor[rgb]{0.13,0.29,0.53}{\textbf{#1}}}
\newcommand{\DataTypeTok}[1]{\textcolor[rgb]{0.13,0.29,0.53}{#1}}
\newcommand{\DecValTok}[1]{\textcolor[rgb]{0.00,0.00,0.81}{#1}}
\newcommand{\DocumentationTok}[1]{\textcolor[rgb]{0.56,0.35,0.01}{\textbf{\textit{#1}}}}
\newcommand{\ErrorTok}[1]{\textcolor[rgb]{0.64,0.00,0.00}{\textbf{#1}}}
\newcommand{\ExtensionTok}[1]{#1}
\newcommand{\FloatTok}[1]{\textcolor[rgb]{0.00,0.00,0.81}{#1}}
\newcommand{\FunctionTok}[1]{\textcolor[rgb]{0.00,0.00,0.00}{#1}}
\newcommand{\ImportTok}[1]{#1}
\newcommand{\InformationTok}[1]{\textcolor[rgb]{0.56,0.35,0.01}{\textbf{\textit{#1}}}}
\newcommand{\KeywordTok}[1]{\textcolor[rgb]{0.13,0.29,0.53}{\textbf{#1}}}
\newcommand{\NormalTok}[1]{#1}
\newcommand{\OperatorTok}[1]{\textcolor[rgb]{0.81,0.36,0.00}{\textbf{#1}}}
\newcommand{\OtherTok}[1]{\textcolor[rgb]{0.56,0.35,0.01}{#1}}
\newcommand{\PreprocessorTok}[1]{\textcolor[rgb]{0.56,0.35,0.01}{\textit{#1}}}
\newcommand{\RegionMarkerTok}[1]{#1}
\newcommand{\SpecialCharTok}[1]{\textcolor[rgb]{0.00,0.00,0.00}{#1}}
\newcommand{\SpecialStringTok}[1]{\textcolor[rgb]{0.31,0.60,0.02}{#1}}
\newcommand{\StringTok}[1]{\textcolor[rgb]{0.31,0.60,0.02}{#1}}
\newcommand{\VariableTok}[1]{\textcolor[rgb]{0.00,0.00,0.00}{#1}}
\newcommand{\VerbatimStringTok}[1]{\textcolor[rgb]{0.31,0.60,0.02}{#1}}
\newcommand{\WarningTok}[1]{\textcolor[rgb]{0.56,0.35,0.01}{\textbf{\textit{#1}}}}
\usepackage{longtable,booktabs,array}
\usepackage{calc} % for calculating minipage widths
\usepackage{caption}
% Make caption package work with longtable
\makeatletter
\def\fnum@table{\tablename~\thetable}
\makeatother
\usepackage{graphicx}
\makeatletter
\def\maxwidth{\ifdim\Gin@nat@width>\linewidth\linewidth\else\Gin@nat@width\fi}
\def\maxheight{\ifdim\Gin@nat@height>\textheight\textheight\else\Gin@nat@height\fi}
\makeatother
% Scale images if necessary, so that they will not overflow the page
% margins by default, and it is still possible to overwrite the defaults
% using explicit options in \includegraphics[width, height, ...]{}
\setkeys{Gin}{width=\maxwidth,height=\maxheight,keepaspectratio}
% Set default figure placement to htbp
\makeatletter
\def\fps@figure{htbp}
\makeatother
\setlength{\emergencystretch}{3em} % prevent overfull lines
\providecommand{\tightlist}{%
  \setlength{\itemsep}{0pt}\setlength{\parskip}{0pt}}
\setcounter{secnumdepth}{-\maxdimen} % remove section numbering
\ifLuaTeX
  \usepackage{selnolig}  % disable illegal ligatures
\fi
\IfFileExists{bookmark.sty}{\usepackage{bookmark}}{\usepackage{hyperref}}
\IfFileExists{xurl.sty}{\usepackage{xurl}}{} % add URL line breaks if available
\urlstyle{same} % disable monospaced font for URLs
\hypersetup{
  pdftitle={Big example},
  hidelinks,
  pdfcreator={LaTeX via pandoc}}

\title{Big example}
\author{}
\date{\vspace{-2.5em}2022-11-17}

\begin{document}
\frame{\titlepage}

\begin{frame}{Data Background}
\protect\hypertarget{data-background}{}
The data was collected using web scrapping sourced from Sports
Reference. The website includes data collected on different players and
sports. Our data in particular was collected on basketball players from
NBA. The data is accurate up-to the spring semester of 2022.
\end{frame}

\begin{frame}{Dictionary}
\protect\hypertarget{dictionary}{}
\begin{longtable}[]{@{}ll@{}}
\toprule()
Column Name & Definition \\
\midrule()
\endhead
Name & Player Name \\
G & Games Played \\
GS & Games Started \\
MP & Minutes Played \\
FG & Field Goals \\
FGA & Field Goals Attempted \\
FG2 & 2 Point Field Goals \\
FG2A & 2 Point Field Goals Attempted \\
FG3 & 3 Point Field Goals \\
FG3A & 3 Point Field Goal Attempted \\
FT & Field Throws \\
FTA & Field Throws Attempted \\
ORB & Offensive Rebounds \\
DRB & Defensive Rebounds \\
\bottomrule()
\end{longtable}
\end{frame}

\begin{frame}{Dictionary (Continued)}
\protect\hypertarget{dictionary-continued}{}
\begin{longtable}[]{@{}ll@{}}
\toprule()
Column Name & Definition \\
\midrule()
\endhead
TRB & Total Rebounds \\
AST & Assists Per Game \\
STL & Steals \\
BLK & Blocks \\
TOV & Turnovers \\
PF & Personal Fouls \\
PTS & Points \\
EFG\_PCT & Effective Field Goal Percentage \\
X\_PER\_G & X Per Game \\
X\_PCT & X Percentage \\
\bottomrule()
\end{longtable}
\end{frame}

\begin{frame}{What Are We Focusing On?}
\protect\hypertarget{what-are-we-focusing-on}{}
As you can see there are a lot of variables in this data set and many
different analyses can be performed. For now we would like to build a
model to predict the number of fouls based on the time each player
spends playing.
\end{frame}

\begin{frame}[fragile]{Preview}
\protect\hypertarget{preview}{}
\begin{Shaded}
\begin{Highlighting}[]
\NormalTok{players }\OtherTok{\textless{}{-}} \FunctionTok{read.csv}\NormalTok{(}\StringTok{"Data/NBA\_players.csv"}\NormalTok{)}
\FunctionTok{head}\NormalTok{(players)[,}\DecValTok{1}\SpecialCharTok{:}\DecValTok{5}\NormalTok{]}
\end{Highlighting}
\end{Shaded}

\begin{verbatim}
##               NAME   G  GS MP_PER_G FG_PER_G
## 1     Álex Abrines 174  16     16.0      1.8
## 2 Precious Achiuwa 134  32     18.4      2.9
## 3       Quincy Acy 337  60     16.0      1.7
## 4     Jaylen Adams  41   1     10.9      1.0
## 5     Steven Adams 664 599     26.8      3.8
## 6      Bam Adebayo 343 239     28.2      5.1
\end{verbatim}
\end{frame}

\begin{frame}[fragile]{First Look}
\protect\hypertarget{first-look}{}
\begin{verbatim}
## Warning: package 'ggplot2' was built under R version 4.2.2
\end{verbatim}

\includegraphics{big_example_files/figure-beamer/unnamed-chunk-3-1.pdf}
\end{frame}

\begin{frame}[fragile]{Creating a Model}
\protect\hypertarget{creating-a-model}{}
There is a built in function that performs smoothing splines in the
`stats' library. A nice thing about the built-in function is its ability
to select the best lambda for us.

\begin{Shaded}
\begin{Highlighting}[]
\FunctionTok{library}\NormalTok{(stats)}
\NormalTok{ss.fit}\FloatTok{.1} \OtherTok{\textless{}{-}} 
  \FunctionTok{smooth.spline}\NormalTok{(players}\SpecialCharTok{$}\NormalTok{MP\_PER\_G, players}\SpecialCharTok{$}\NormalTok{PF\_PER\_G)}
\end{Highlighting}
\end{Shaded}

If we were doing it ``by hand'' it would look like this

\begin{Shaded}
\begin{Highlighting}[]
\CommentTok{\# This will be used for manual model creation.}
\end{Highlighting}
\end{Shaded}
\end{frame}

\begin{frame}{Smoothing Spline Plot}
\protect\hypertarget{smoothing-spline-plot}{}
\includegraphics{big_example_files/figure-beamer/unnamed-chunk-6-1.pdf}
\end{frame}

\begin{frame}[fragile]{Changing our lambda}
\protect\hypertarget{changing-our-lambda}{}
\begin{Shaded}
\begin{Highlighting}[]
\FunctionTok{library}\NormalTok{(stats)}
\NormalTok{ss.fit}\FloatTok{.2} \OtherTok{\textless{}{-}} 
  \FunctionTok{smooth.spline}\NormalTok{(players}\SpecialCharTok{$}\NormalTok{MP\_PER\_G, }
\NormalTok{                players}\SpecialCharTok{$}\NormalTok{PF\_PER\_G, }
                \AttributeTok{lambda=}\DecValTok{0}\NormalTok{)}

\NormalTok{ss.fit}\FloatTok{.3} \OtherTok{\textless{}{-}} 
  \FunctionTok{smooth.spline}\NormalTok{(players}\SpecialCharTok{$}\NormalTok{MP\_PER\_G, }
\NormalTok{                players}\SpecialCharTok{$}\NormalTok{PF\_PER\_G, }
                \AttributeTok{lambda=}\DecValTok{10000}\NormalTok{)}
\end{Highlighting}
\end{Shaded}
\end{frame}

\begin{frame}[fragile]{Comparing Lambdas}
\protect\hypertarget{comparing-lambdas}{}
\begin{verbatim}
## Warning: package 'gridExtra' was built under R version 4.2.2
\end{verbatim}

\includegraphics{big_example_files/figure-beamer/unnamed-chunk-8-1.pdf}
\end{frame}

\end{document}
